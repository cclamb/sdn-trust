\documentclass[12pt,letterpaper]{article}
\usepackage[utf8]{inputenc}
\usepackage{amsmath}
\usepackage{amsfonts}
\usepackage{amssymb}
\usepackage{graphicx}
\usepackage{fourier}
\usepackage{xcolor}
\author{Christopher C. Lamb}
\begin{document}
\section{Title}

\section{Networks in the Bad Old Days}
\begin{itemize}
\item Large networks
\item many devices needed individual management
\item configuration version control unheard of (well, mostly)
\item expensive to manage and easy to mess up
\end{itemize}

\section{The Brave New World!}
\begin{itemize}
\item Centralized management
\item more responsive potentially
\item easy (-ier) to manage
\item potentially more responsive
\item applications, controllers, repositories combine to provide better infrastructure control
\end{itemize}

\section{Brave New implications}
(see slides)

\newpage
\section{host $\Longleftrightarrow$ switch}
\begin{itemize}
\item {\color{orange} Confidentiality not critical}
	\begin{itemize}
	\item switches don't consume host services
	\item hosts don't confirm delivery by a specific switch (usually)
	\item no {\sl explicit} control traffic
	\end{itemize}
\item {\color{red} Integrity vital}
	\begin{itemize}
	\item hosts expect reliable data plane information transfer
	\item switches don't communicate to hosts
	\end{itemize}
\item {\color{orange} Availability expected} 
	\begin{itemize}
	\item Hosts do want networks up when they need them, but a data plane concern
	\end{itemize}
\item {\color{green} Non-repudiation}
	\begin{itemize}
	\item Hosts don't expect non-repudiation of traffic handling
	\item Switches don't handle non-repudiation of handled traffic (though other devices might to this, forensics, auditing, etc.)
	\end{itemize}
\item {\color{green} Authentication}
	\begin{itemize}
	\item Hosts generally don't need to authenticate a switch, though with very sensitive traffic this may be important
	\item Switches may authenticate hosts when doing data plane traffic prioritization or service shaping
	\end{itemize}
\end{itemize}

\newpage
\section{switch $\Longleftrightarrow$ controller}
\begin{itemize}
\item {\color{orange} Confidentiality only important in some edge cases}
	\begin{itemize}
	\item Not generally important as the effects of any controller messages is visible
	\item Defense-oriented messages (e.g. related to malicious traffic redirection) may need to be confidential; this implies all traffic kept confidential in these cases
	\end{itemize}
\item {\color{red} Integrity again vital}
	\begin{itemize}
	\item Switches need to be able to trust controller messages
	\item Controllers will very likely issue commands to specific switches based on switch status, so messages from switches to controllers must be trustworthy
	\end{itemize}
\item {\color{red} Availability paramount} 
	\begin{itemize}
	\item Controllers must be available to switches for SDN to work correctly; otherwise, switch behavior is undefined, though they will usually use the most recent flow table for a while
	\item Switch availability is much less important from a control-plane perspective
	\end{itemize}
\item {\color{orange} Non-repudiation perhaps more important}
	\begin{itemize}
	\item Useful for forensics, network debugging, to see which controllers issued which commands and past switch status
	\item Useful for trust measures
	\end{itemize}
\item {\color{red} Authentication of controllers important}
	\begin{itemize}
	\item Switches should be able to authenticate a given controller to establish controller authority
	\item Controllers may only need to authenciate switches in high security environments, to assure that all managed switches are in fact authorized to handle traffic on a network
	\end{itemize}
\end{itemize}

\newpage
\section{controller $\Longleftrightarrow$ repository}
Important to note, this model can be implemented via application access to controllers as well
\begin{itemize}
\item {\color{orange} Confidentiality not always vital}
	\begin{itemize}
	\item Controllers may need to access repositories for global state information
	\item Repositories may send information to controllers
	\item In either case, the information can generally be derived from other sources
	\item Security-related information may be important in some cases
	\end{itemize}
\item {\color{red} Integrity important, as usual}
	\begin{itemize}
	\item Controllers need to be able to trust actionable information from repositories
	\item Repositories generally have no data dependencies on controllers, though they may in cases where controllers report status to enable global network awareness
	\end{itemize}
\item {\color{green} Availability not vital} 
	\begin{itemize}
	\item Controllers can make local decisions that can aggregate into a nearly optimal global state
	\item Global state information becomes more important at scale as the cost of cumulative inefficiency grows
	\end{itemize}
\item {\color{green} Non-repudiation a big less important for core control plane functions}
	\begin{itemize}
	\item Again most useful for trust evaluation and forensics
	\end{itemize}
\item {\color{red} Authentication of repositories important}
	\begin{itemize}
	\item If a controller is going to use repository supplied information to make control decisions, the source of that information must be trustworthy
	\item Likewise, repositories collecting information from controllers (or other sources) must be able to have confidence that the information delivered is from a source that can be trusted
	\end{itemize}
\end{itemize}

\newpage
\section{controller $\Longleftrightarrow$ application}
Applications are generalities of the previous repository construct, but run into issues when we have multiple applications accessing a given controller
\begin{itemize}
\item {\color{orange} Confidentiality based on application}
	\begin{itemize}
	\item Confidentiality requirements are really based on the application type and the information submitted to the controller; e.g. logging apps may not be as important as security apps
	\end{itemize}
\item {\color{red} Integrity important, as usual}
	\begin{itemize}
	\item Controllers need to be able to trust actionable information, and applications need to be able to trust controllers for data collection
	\end{itemize}
\item {\color{green} Availability not vital} 
	\begin{itemize}
	\item Heavily dependent on the specific application
	\end{itemize}
\item {\color{green} Non-repudiation needs again based on application}
	\begin{itemize}
	\item Most useful for trust evaluation and forensics
	\end{itemize}
\item {\color{red} Authentication of certain applications important}
	\begin{itemize}
	\item If an app can influence a controller, it must be authenticated
	\item If a controller submits data to an app, it must be authenticated
	\end{itemize}
\end{itemize}

\newpage
\section{Attribute Commonality}
\begin{itemize}
\item {\color{green} Confidentiality}
	\begin{itemize}
	\item Usually information can be derived from network behavior
	\item Exceptions for cyber-security use cases; these can drive confidentiality needs into entire system
	\end{itemize}
\item {\color{red} Integrity}
	\begin{itemize}
	\item Actionable information leading to control-plane decisions needs to be trustworthy
	\end{itemize}
\item {\color{orange} Availability}
	\begin{itemize}
	\item Of varying importance
	\item Most important in switch $\Longleftrightarrow$ controller relationships
	\end{itemize}
\item {\color{green} Non-repudiation}
	\begin{itemize}
	\item Not that important, usually relegated to trust or forensics use cases
	\item Cheap to add if we already have authentication and message integrity though
	\end{itemize}
\item {\color{red} Authentication}
	\begin{itemize}
	\item Related to integrity
	\item Control agents (controllers, etc.) need to be able to establish that a source is trustworthy as well as that the information from that source has not been tampered with
	\end{itemize}
\end{itemize}

\newpage
\section{Differentiating Attributes of SDN}
Compared to other more agent-centric systems, SDN control systems have some advantages:
\begin{itemize}
\item Limited control-plane volatility
	\begin{itemize}
	\item MANETs and agent-based systems are much more chaotic with respect to functional distribution (many devices wear multiple commmunication hats) and suffer from frequent attach / detach issues
	\end{itemize}
\item Centralized High-Availability
	\begin{itemize}
	\item Any high-availability requirements are constrained to specific functional areas (e.g. controllers)
	\end{itemize}
\item Clearly Defined Roles
	\begin{itemize}
	\item SDN entities have clear roles; systems in MANETs or agent-based systems frequently do not
	\end{itemize}
\item Predicable Expected Behavior
	\begin{itemize}
	\item Clearly defined roles should lead to more predicable behavior and correspondingly easier behavioral outlier detection
	\end{itemize}
\end{itemize}

\section{Comments and Questions}
(see slides)


\end{document}